% Options for packages loaded elsewhere
% Options for packages loaded elsewhere
\PassOptionsToPackage{unicode}{hyperref}
\PassOptionsToPackage{hyphens}{url}
\PassOptionsToPackage{dvipsnames,svgnames,x11names}{xcolor}
%
\documentclass[
]{article}
\usepackage{xcolor}
\usepackage{amsmath,amssymb}
\setcounter{secnumdepth}{5}
\usepackage{iftex}
\ifPDFTeX
  \usepackage[T1]{fontenc}
  \usepackage[utf8]{inputenc}
  \usepackage{textcomp} % provide euro and other symbols
\else % if luatex or xetex
  \usepackage{unicode-math} % this also loads fontspec
  \defaultfontfeatures{Scale=MatchLowercase}
  \defaultfontfeatures[\rmfamily]{Ligatures=TeX,Scale=1}
\fi
\usepackage{lmodern}
\ifPDFTeX\else
  % xetex/luatex font selection
\fi
% Use upquote if available, for straight quotes in verbatim environments
\IfFileExists{upquote.sty}{\usepackage{upquote}}{}
\IfFileExists{microtype.sty}{% use microtype if available
  \usepackage[]{microtype}
  \UseMicrotypeSet[protrusion]{basicmath} % disable protrusion for tt fonts
}{}
\makeatletter
\@ifundefined{KOMAClassName}{% if non-KOMA class
  \IfFileExists{parskip.sty}{%
    \usepackage{parskip}
  }{% else
    \setlength{\parindent}{0pt}
    \setlength{\parskip}{6pt plus 2pt minus 1pt}}
}{% if KOMA class
  \KOMAoptions{parskip=half}}
\makeatother
% Make \paragraph and \subparagraph free-standing
\makeatletter
\ifx\paragraph\undefined\else
  \let\oldparagraph\paragraph
  \renewcommand{\paragraph}{
    \@ifstar
      \xxxParagraphStar
      \xxxParagraphNoStar
  }
  \newcommand{\xxxParagraphStar}[1]{\oldparagraph*{#1}\mbox{}}
  \newcommand{\xxxParagraphNoStar}[1]{\oldparagraph{#1}\mbox{}}
\fi
\ifx\subparagraph\undefined\else
  \let\oldsubparagraph\subparagraph
  \renewcommand{\subparagraph}{
    \@ifstar
      \xxxSubParagraphStar
      \xxxSubParagraphNoStar
  }
  \newcommand{\xxxSubParagraphStar}[1]{\oldsubparagraph*{#1}\mbox{}}
  \newcommand{\xxxSubParagraphNoStar}[1]{\oldsubparagraph{#1}\mbox{}}
\fi
\makeatother

\usepackage{color}
\usepackage{fancyvrb}
\newcommand{\VerbBar}{|}
\newcommand{\VERB}{\Verb[commandchars=\\\{\}]}
\DefineVerbatimEnvironment{Highlighting}{Verbatim}{commandchars=\\\{\}}
% Add ',fontsize=\small' for more characters per line
\usepackage{framed}
\definecolor{shadecolor}{RGB}{241,243,245}
\newenvironment{Shaded}{\begin{snugshade}}{\end{snugshade}}
\newcommand{\AlertTok}[1]{\textcolor[rgb]{0.68,0.00,0.00}{#1}}
\newcommand{\AnnotationTok}[1]{\textcolor[rgb]{0.37,0.37,0.37}{#1}}
\newcommand{\AttributeTok}[1]{\textcolor[rgb]{0.40,0.45,0.13}{#1}}
\newcommand{\BaseNTok}[1]{\textcolor[rgb]{0.68,0.00,0.00}{#1}}
\newcommand{\BuiltInTok}[1]{\textcolor[rgb]{0.00,0.23,0.31}{#1}}
\newcommand{\CharTok}[1]{\textcolor[rgb]{0.13,0.47,0.30}{#1}}
\newcommand{\CommentTok}[1]{\textcolor[rgb]{0.37,0.37,0.37}{#1}}
\newcommand{\CommentVarTok}[1]{\textcolor[rgb]{0.37,0.37,0.37}{\textit{#1}}}
\newcommand{\ConstantTok}[1]{\textcolor[rgb]{0.56,0.35,0.01}{#1}}
\newcommand{\ControlFlowTok}[1]{\textcolor[rgb]{0.00,0.23,0.31}{\textbf{#1}}}
\newcommand{\DataTypeTok}[1]{\textcolor[rgb]{0.68,0.00,0.00}{#1}}
\newcommand{\DecValTok}[1]{\textcolor[rgb]{0.68,0.00,0.00}{#1}}
\newcommand{\DocumentationTok}[1]{\textcolor[rgb]{0.37,0.37,0.37}{\textit{#1}}}
\newcommand{\ErrorTok}[1]{\textcolor[rgb]{0.68,0.00,0.00}{#1}}
\newcommand{\ExtensionTok}[1]{\textcolor[rgb]{0.00,0.23,0.31}{#1}}
\newcommand{\FloatTok}[1]{\textcolor[rgb]{0.68,0.00,0.00}{#1}}
\newcommand{\FunctionTok}[1]{\textcolor[rgb]{0.28,0.35,0.67}{#1}}
\newcommand{\ImportTok}[1]{\textcolor[rgb]{0.00,0.46,0.62}{#1}}
\newcommand{\InformationTok}[1]{\textcolor[rgb]{0.37,0.37,0.37}{#1}}
\newcommand{\KeywordTok}[1]{\textcolor[rgb]{0.00,0.23,0.31}{\textbf{#1}}}
\newcommand{\NormalTok}[1]{\textcolor[rgb]{0.00,0.23,0.31}{#1}}
\newcommand{\OperatorTok}[1]{\textcolor[rgb]{0.37,0.37,0.37}{#1}}
\newcommand{\OtherTok}[1]{\textcolor[rgb]{0.00,0.23,0.31}{#1}}
\newcommand{\PreprocessorTok}[1]{\textcolor[rgb]{0.68,0.00,0.00}{#1}}
\newcommand{\RegionMarkerTok}[1]{\textcolor[rgb]{0.00,0.23,0.31}{#1}}
\newcommand{\SpecialCharTok}[1]{\textcolor[rgb]{0.37,0.37,0.37}{#1}}
\newcommand{\SpecialStringTok}[1]{\textcolor[rgb]{0.13,0.47,0.30}{#1}}
\newcommand{\StringTok}[1]{\textcolor[rgb]{0.13,0.47,0.30}{#1}}
\newcommand{\VariableTok}[1]{\textcolor[rgb]{0.07,0.07,0.07}{#1}}
\newcommand{\VerbatimStringTok}[1]{\textcolor[rgb]{0.13,0.47,0.30}{#1}}
\newcommand{\WarningTok}[1]{\textcolor[rgb]{0.37,0.37,0.37}{\textit{#1}}}

\usepackage{longtable,booktabs,array}
\usepackage{calc} % for calculating minipage widths
% Correct order of tables after \paragraph or \subparagraph
\usepackage{etoolbox}
\makeatletter
\patchcmd\longtable{\par}{\if@noskipsec\mbox{}\fi\par}{}{}
\makeatother
% Allow footnotes in longtable head/foot
\IfFileExists{footnotehyper.sty}{\usepackage{footnotehyper}}{\usepackage{footnote}}
\makesavenoteenv{longtable}
\usepackage{graphicx}
\makeatletter
\newsavebox\pandoc@box
\newcommand*\pandocbounded[1]{% scales image to fit in text height/width
  \sbox\pandoc@box{#1}%
  \Gscale@div\@tempa{\textheight}{\dimexpr\ht\pandoc@box+\dp\pandoc@box\relax}%
  \Gscale@div\@tempb{\linewidth}{\wd\pandoc@box}%
  \ifdim\@tempb\p@<\@tempa\p@\let\@tempa\@tempb\fi% select the smaller of both
  \ifdim\@tempa\p@<\p@\scalebox{\@tempa}{\usebox\pandoc@box}%
  \else\usebox{\pandoc@box}%
  \fi%
}
% Set default figure placement to htbp
\def\fps@figure{htbp}
\makeatother


% definitions for citeproc citations
\NewDocumentCommand\citeproctext{}{}
\NewDocumentCommand\citeproc{mm}{%
  \begingroup\def\citeproctext{#2}\cite{#1}\endgroup}
\makeatletter
 % allow citations to break across lines
 \let\@cite@ofmt\@firstofone
 % avoid brackets around text for \cite:
 \def\@biblabel#1{}
 \def\@cite#1#2{{#1\if@tempswa , #2\fi}}
\makeatother
\newlength{\cslhangindent}
\setlength{\cslhangindent}{1.5em}
\newlength{\csllabelwidth}
\setlength{\csllabelwidth}{3em}
\newenvironment{CSLReferences}[2] % #1 hanging-indent, #2 entry-spacing
 {\begin{list}{}{%
  \setlength{\itemindent}{0pt}
  \setlength{\leftmargin}{0pt}
  \setlength{\parsep}{0pt}
  % turn on hanging indent if param 1 is 1
  \ifodd #1
   \setlength{\leftmargin}{\cslhangindent}
   \setlength{\itemindent}{-1\cslhangindent}
  \fi
  % set entry spacing
  \setlength{\itemsep}{#2\baselineskip}}}
 {\end{list}}
\usepackage{calc}
\newcommand{\CSLBlock}[1]{\hfill\break\parbox[t]{\linewidth}{\strut\ignorespaces#1\strut}}
\newcommand{\CSLLeftMargin}[1]{\parbox[t]{\csllabelwidth}{\strut#1\strut}}
\newcommand{\CSLRightInline}[1]{\parbox[t]{\linewidth - \csllabelwidth}{\strut#1\strut}}
\newcommand{\CSLIndent}[1]{\hspace{\cslhangindent}#1}



\setlength{\emergencystretch}{3em} % prevent overfull lines

\providecommand{\tightlist}{%
  \setlength{\itemsep}{0pt}\setlength{\parskip}{0pt}}



 


\makeatletter
\@ifpackageloaded{caption}{}{\usepackage{caption}}
\AtBeginDocument{%
\ifdefined\contentsname
  \renewcommand*\contentsname{Table of contents}
\else
  \newcommand\contentsname{Table of contents}
\fi
\ifdefined\listfigurename
  \renewcommand*\listfigurename{List of Figures}
\else
  \newcommand\listfigurename{List of Figures}
\fi
\ifdefined\listtablename
  \renewcommand*\listtablename{List of Tables}
\else
  \newcommand\listtablename{List of Tables}
\fi
\ifdefined\figurename
  \renewcommand*\figurename{Figure}
\else
  \newcommand\figurename{Figure}
\fi
\ifdefined\tablename
  \renewcommand*\tablename{Table}
\else
  \newcommand\tablename{Table}
\fi
}
\@ifpackageloaded{float}{}{\usepackage{float}}
\floatstyle{ruled}
\@ifundefined{c@chapter}{\newfloat{codelisting}{h}{lop}}{\newfloat{codelisting}{h}{lop}[chapter]}
\floatname{codelisting}{Listing}
\newcommand*\listoflistings{\listof{codelisting}{List of Listings}}
\makeatother
\makeatletter
\makeatother
\makeatletter
\@ifpackageloaded{caption}{}{\usepackage{caption}}
\@ifpackageloaded{subcaption}{}{\usepackage{subcaption}}
\makeatother
\usepackage{bookmark}
\IfFileExists{xurl.sty}{\usepackage{xurl}}{} % add URL line breaks if available
\urlstyle{same}
\hypersetup{
  pdftitle={Modelos Paramétricos},
  pdfauthor={Sergio M. Nava Muñoz},
  colorlinks=true,
  linkcolor={blue},
  filecolor={Maroon},
  citecolor={Blue},
  urlcolor={Blue},
  pdfcreator={LaTeX via pandoc}}


\title{Modelos Paramétricos}
\author{Sergio M. Nava Muñoz}
\date{2025-06-01}
\begin{document}
\maketitle

\renewcommand*\contentsname{Table of contents}
{
\hypersetup{linkcolor=}
\setcounter{tocdepth}{2}
\tableofcontents
}

\section{Objetivo}\label{objetivo}

\begin{itemize}
\tightlist
\item
  Introducir el concepto de máxima verosimilitud (MLE)
\item
  Aplicar MLE a modelos de tiempo de supervivencia
\item
  Usar R para estimar parámetros en presencia de censura
\item
  Interpretar estimaciones y su relación con funciones de supervivencia
\end{itemize}

\begin{center}\rule{0.5\linewidth}{0.5pt}\end{center}

\section{¿Qué es la verosimilitud?}\label{quuxe9-es-la-verosimilitud}

\begin{itemize}
\item
  Es una función que mide \textbf{cuán probable} es observar los datos
  dados ciertos parámetros. Este enfoque es introducido en Moore (2016)
  como base para la estimación paramétrica en supervivencia.
\item
  Dado un modelo con función de densidad \(f(t; \theta)\), la
  \textbf{verosimilitud} para un conjunto de datos \(t_1, ..., t_n\) es:
\end{itemize}

\[
L(\theta) = \prod_{i=1}^n f(t_i; \theta)
\]

\begin{itemize}
\tightlist
\item
  Se busca el valor \(\hat{\theta}\) que \textbf{maximiza} \(L(\theta)\)
  o, más comúnmente, \(\log L(\theta)\)
\end{itemize}

\begin{center}\rule{0.5\linewidth}{0.5pt}\end{center}

\section{Caso con censura}\label{caso-con-censura}

\begin{itemize}
\item
  Si hay censura, se observa:

  \begin{itemize}
  \tightlist
  \item
    Tiempo \(t_i\)
  \item
    Indicador \(\delta_i = 1\) si ocurrió el evento, \(0\) si censurado
  \end{itemize}
\item
  La función de verosimilitud se ajusta:
\end{itemize}

\[
L(\theta) = \prod_{i=1}^n [f(t_i; \theta)]^{\delta_i} [S(t_i; \theta)]^{1 - \delta_i}
\]

Este desarrollo puede encontrarse también en Klein \& Moeschberger
(2003) como parte de la teoría general de modelos paramétricos en
supervivencia.

\begin{center}\rule{0.5\linewidth}{0.5pt}\end{center}

\section{Ejemplo: distribución
exponencial}\label{ejemplo-distribuciuxf3n-exponencial}

\begin{itemize}
\item
  Supón \(T \sim \text{Exp}(\lambda)\), entonces:

  \begin{itemize}
  \tightlist
  \item
    \(f(t) = \lambda e^{-\lambda t}\)
  \item
    \(S(t) = e^{-\lambda t}\)
  \end{itemize}
\item
  Verosimilitud con censura:
\end{itemize}

\[
L(\lambda) = \prod_{i=1}^n [\lambda e^{-\lambda t_i}]^{\delta_i} [e^{-\lambda t_i}]^{1 - \delta_i}
= \lambda^d e^{-\lambda \sum t_i}
\]

\begin{itemize}
\tightlist
\item
  \(d = \sum \delta_i\), número de eventos
\end{itemize}

\begin{center}\rule{0.5\linewidth}{0.5pt}\end{center}

\section{Derivación del estimador
MLE}\label{derivaciuxf3n-del-estimador-mle}

Para una discusión general sobre el principio de máxima verosimilitud,
véase Casella \& Berger (2002)

\begin{itemize}
\tightlist
\item
  Log-verosimilitud:
\end{itemize}

\[
\ell(\lambda) = d \log \lambda - \lambda \sum t_i
\]

\begin{itemize}
\tightlist
\item
  Derivando e igualando a 0:
\end{itemize}

\[
\frac{d}{d\lambda} \ell(\lambda) = \frac{d}{\lambda} - \sum t_i = 0
\Rightarrow \hat{\lambda} = \frac{d}{\sum t_i}
\]

\begin{center}\rule{0.5\linewidth}{0.5pt}\end{center}

\section{Código R: estimación con
censura}\label{cuxf3digo-r-estimaciuxf3n-con-censura}

\begin{verbatim}
[1] 0.1
\end{verbatim}

\begin{center}\rule{0.5\linewidth}{0.5pt}\end{center}

\section{\texorpdfstring{Comparación con
\texttt{survreg}}{Comparación con survreg}}\label{comparaciuxf3n-con-survreg}

\begin{verbatim}

Call:
survreg(formula = Surv(tt, status) ~ 1, dist = "exponential")
            Value Std. Error    z       p
(Intercept) 2.303      0.577 3.99 6.7e-05

Scale fixed at 1 

Exponential distribution
Loglik(model)= -9.9   Loglik(intercept only)= -9.9
Number of Newton-Raphson Iterations: 4 
n= 6 
\end{verbatim}

\begin{itemize}
\tightlist
\item
  La estimación \(\hat{\lambda}\) se relaciona con \(\text{scale}^{-1}\)
\end{itemize}

\section{\texorpdfstring{¿Dónde está \(\hat{\lambda}\) en
\texttt{survreg()}?}{¿Dónde está \textbackslash hat\{\textbackslash lambda\} en survreg()?}}\label{duxf3nde-estuxe1-hatlambda-en-survreg}

\begin{Shaded}
\begin{Highlighting}[]
\NormalTok{fit }\OtherTok{\textless{}{-}} \FunctionTok{survreg}\NormalTok{(}\FunctionTok{Surv}\NormalTok{(tt, status) }\SpecialCharTok{\textasciitilde{}} \DecValTok{1}\NormalTok{, }\AttributeTok{dist =} \StringTok{"exponential"}\NormalTok{)}
\FunctionTok{summary}\NormalTok{(fit)}
\end{Highlighting}
\end{Shaded}

\begin{itemize}
\item
  El modelo AFT estima: \[
  \log(T) = \mu + \varepsilon, \quad \text{con } \mu = \text{Intercepto}
  \]
\item
  Para la distribución exponencial: \[
  \mu = \log\left(\frac{1}{\lambda}\right) \Rightarrow
  \hat{\lambda} = e^{-\mu}
  \]
\end{itemize}

\begin{verbatim}
(Intercept) 
        0.1 
\end{verbatim}

\begin{itemize}
\tightlist
\item
  En tu salida: \texttt{Intercept\ =\ 2.303}\\
  Entonces: \(\hat{\lambda} = e^{-2.303} \approx 0.1\)
\end{itemize}

\begin{center}\rule{0.5\linewidth}{0.5pt}\end{center}

\section{Interpretación de
resultados}\label{interpretaciuxf3n-de-resultados}

\begin{itemize}
\tightlist
\item
  \(\hat{\lambda}\) es la tasa de riesgo constante estimada
\item
  Su inverso es la \textbf{media de supervivencia}:
\end{itemize}

\[
\hat{E}(T) = \frac{1}{\hat{\lambda}}
\]

\begin{center}\rule{0.5\linewidth}{0.5pt}\end{center}

\section{Actividad práctica}\label{actividad-pruxe1ctica}

\begin{enumerate}
\def\labelenumi{\arabic{enumi}.}
\tightlist
\item
  Simula un conjunto de datos de supervivencia con censura
\item
  Calcula el estimador de máxima verosimilitud para \(\lambda\)
\item
  Usa \texttt{survreg} para confirmar
\end{enumerate}

\begin{center}\rule{0.5\linewidth}{0.5pt}\end{center}

\section{Conclusiones}\label{conclusiones}

\begin{itemize}
\tightlist
\item
  MLE permite incorporar eventos y censura de forma natural
\item
  Las expresiones son simples en modelos paramétricos como el
  exponencial
\item
  Herramientas de R hacen este proceso accesible
\end{itemize}

\begin{center}\rule{0.5\linewidth}{0.5pt}\end{center}

\section*{Lecturas recomendadas}\label{lecturas-recomendadas}
\addcontentsline{toc}{section}{Lecturas recomendadas}

\phantomsection\label{refs}
\begin{CSLReferences}{1}{0}
\bibitem[\citeproctext]{ref-casella2002statistical}
Casella, G., \& Berger, R. L. (2002). \emph{Statistical inference} (2nd
ed.). Duxbury.

\bibitem[\citeproctext]{ref-klein2003}
Klein, J. P., \& Moeschberger, M. L. (2003). \emph{Survival analysis:
Techniques for censored and truncated data} (2nd ed.). Springer.

\bibitem[\citeproctext]{ref-moore2016}
Moore, D. F. (2016). \emph{Applied survival analysis using r} (2nd ed.).
Springer. \url{https://doi.org/10.1007/978-3-319-31245-3}

\end{CSLReferences}




\end{document}
